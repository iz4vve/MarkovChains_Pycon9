\documentclass[11pt]{beamer}
\usepackage{listings}
\usetheme{Berkeley}
\usecolortheme{seagull}
\title[Python Course]{\textbf{Python - (Slightly) advanced course}}
\institute[Optum]{Optum Technology}
\author[]{Pietro Mascolo}
\date{2017}
\logo{\includegraphics[height=1.6cm]{imgs/optum_logo.png}}

\definecolor{optum-orange}{RGB}{232, 119, 34}
\definecolor{optum-green}{RGB}{7,133,118}

% colours for beamer http://www.cpt.univ-mrs.fr/~masson/latex/Beamer-appearance-cheat-sheet.pdf
\setbeamercolor*{palette primary}{use=structure,fg=white,bg=optum-orange}
\setbeamercolor*{palette sidebar quaternary}{fg=white,bg=optum-orange}
\setbeamercolor*{block title}{fg=white,bg=optum-orange}

\setbeamercolor*{palette sidebar secondary}{use=white,fg=black,bg=optum-green}
\setbeamercolor*{sidebar}{fg=white,bg=optum-orange}




\begin{document}

\maketitle

\section{Introduction}
\begin{frame}
\frametitle{Python}
Keep calm and `import this`
\end{frame}


\section{Warm-up}
\begin{frame}
\frametitle{A little exercise to get started...}
\lstinputlisting{warmup.py}
\end{frame}

\section{Iterators and Generators}
\begin{frame}
	\frametitle{Iterators and generators}
	\only<1>{
	\begin{figure}
		\includegraphics[width=\textwidth]{imgs/iterable-vs-iterator.png}
		\caption{Iterable vs iterator}
	\end{figure}
	}
	\only<2>{
	\begin{figure}
		\includegraphics[width=\textwidth]{imgs/relationships.png}
		\caption{Containers and iterables}
	\end{figure}
	}
	\only<3>{

	\begin{block} {When is this useful?}
		{\lstinputlisting{iterators.py}}
	\end{block}

	}
\end{frame}

\section{Exercises}
\begin{frame}
\frametitle{\texttt{lambda, map, filter} and functional aspects}
\begin{block}
{Functional aspects}{\texttt{GOTO code}}
\end{block}
\end{frame}

\section{Data wrangling}
\begin{frame}
	\frametitle{Handling data}
	\only<1>{
		\begin{block}{Useful libraries}
			{
				\begin{itemize}
					\item \texttt{numpy} - linear algebra;
					\item \texttt{scipy} - scientific computing;
					\item \texttt{statstools} - statistics;
					\item \texttt{pandas} - data structures;
					\item \texttt{scikit-learn} - machine learning algorithms.
				\end{itemize}
			}
		\end{block}
	}
\end{frame}

\section{Advanced OOP}
\begin{frame}
\frametitle{Advanced OOP concepts}
	\only<1>{
		\begin{block}{Useful concepts}
			{
				\begin{itemize}
					\item abstract classes;
					\item composition and inheritance;
					\item multiple inheritance;
					\item \texttt{super}.
				\end{itemize}
			}
		\end{block}
	}
\end{frame}


%\section{Testing}
%\begin{frame}
%\frametitle{Testing}
%\end{frame}

%\section{Coroutines}
%\begin{frame}
%\frametitle{Coroutines}
%\end{frame}

\section{Conclusion}
\begin{frame}
\begin{block}{Thanks!}{Q/A time!}
\end{block}
\end{frame}


\end{document}
